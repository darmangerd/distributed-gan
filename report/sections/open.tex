\chapter{A note on the Open Science principles}
Our project utilized open-source software, ensuring that all code and development environments are accessible to the research community and beyond. We developed our MD-GAN using only open-source dependencies to facilitate ease of use and adaptation by other researchers. This approach promotes transparency, collaboration, and reproducibility in research, allowing others to build upon and extend our work.

Included within these dependencies are the publicly available datasets we used to produce our results: MNIST, CelebA, and CIFAR-10. These datasets are open and widely used, forming the core of our experiments and allowing us to benchmark our model's performance against established standards. By detailing our data usage and experimental setups in our documentation, we ensure that other researchers can replicate our studies or extend them with new data under similar conditions.

The entire implementation of our GAN model and all the tools to reproduce our results, including the architecture and algorithms, are available on GitHub\footnote{\url{https://github.com/darmangerd/distributed-gan}}.

A key aspect of our project's alignment with open science is our commitment to reproducibility. We have documented all experimental procedures, model configurations, and hyperparameter settings. This comprehensive documentation ensures that others can reproduce our results and verify our claims. Additionally, by reproducing the results from the foundational MD-GAN paper, we contribute to the validation of previous findings within the community.